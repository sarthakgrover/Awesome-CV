%-------------------------------------------------------------------------------
%	SECTION TITLE
%-------------------------------------------------------------------------------
\cvsection{Work Experience}

%-------------------------------------------------------------------------------
%	CONTENT
%-------------------------------------------------------------------------------
\begin{cventries}

%---------------------------------------------------------
  \cventry
    {Manager} % Job title
    {Sugandhaa Co., Ltd.} % Organization
    {New Delhi, INDIA} % Location
    {Apr. 2018 - Present} % Date(s)
    {
      \begin{cvitems} % Description(s) of tasks/responsibilities
      \item {Leading a family-owned business. Responsible for inventory, account management, and conducting sales.}
      \item {Proficient in interacting directly with customers and managing daily finances to ensure smooth business operations.}
      \end{cvitems}
    }

%---------------------------------------------------------
  \cventry
    {Freelance Researcher} % Job title
    {} % Organization
    {} % Location
    {Dec. 2018 - Present} % Date(s)
    {
      \begin{cvitems} % Description(s) of tasks/responsibilities
      \item {\textbf{Network anomaly detection:} Designed and implemented an anomaly detection algorithm for NetFlow records.
      		\link{https://github.com/shahifaqeer/netflow-anomaly-detector}}
      \item {Used machine learning to extract and transform packet features and detect anomalous behavior in real-time.}
      \item {Successfully identified port scans and DoS activity using statistical, information-theoretic, and clustering-based approaches.}
%      
      \item {\textbf{CDN performance analysis:} Conducted website measurements and tested performance of web page load times.
      		\link{https://github.com/shahifaqeer/cdn-analysis}}
      \item {Developed a scalable rule-based CDN estimation algorithm for websites using xcache, whois, parsing, and DNS information.}
      \item {Revealed that for some websites, performance can be substantially improved using CDNs offered by their own ASNs.}
%      
      \item {\textbf{Coursework:} Awarded certificate for Neural Networks and Deep Learning by deeplearning.ai on Coursera.
      		\link{https://www.coursera.org/account/accomplishments/certificate/M6LHSEZJT4X6?utm_medium=certificate&utm_source=link&utm_campaign=copybutton_certificate}}
      \item {Awarded certificate for Functional Programming Principles in Scala by  EPFL on Coursera.
      		\link{https://www.coursera.org/account/accomplishments/certificate/V2LQXVYEBCJ4?utm_medium=certificate&utm_source=link&utm_campaign=copybutton_certificate}}
      		%TODO consider moving to accomplishments
      \end{cvitems}
    }

%---------------------------------------------------------
  \cventry
    {Graduate Research Assistant} % Job title
    {Princeton University} % Organization
    {Princeton, NJ, USA} % Location
    {Jan. 2015 - Sep. 2017} % Date(s)
    {
      \begin{cvitems} % Description(s) of tasks/responsibilities
      \item {Advisers: Prof. Nick Feamster (University of Chicago) and Dr. Roya Ensafi (University of Michigan)}
%      
      \item {\textbf{Identifying Internet of Things (IoTs):} Captured and analyzed packets from various IoT devices in the lab.
      		\link{https://github.com/shahifaqeer/iot-dns}}
      \item {Used machine learning to extract and transform (FFT, PCA, MMT, KDE) volumetric and distribution features from packet headers.}
      \item {Used clustering (kmeans, Spectral, MeanShift, DBSCAN) to identify IoTs, and distinguish activity and background traffic.}
      \item {Built an IoT behavior monitor and demonstrated that it successfully detects DDoS attack anomalies using DBSCAN.
      		\link{https://github.com/shahifaqeer/IoTAnomalyDetection}}
%      
      \item {\textbf{Real-time IoT query system:} Developed a filtering algorithm to discard non-IoT network traffic using DNS queries.
      		\link{https://github.com/shahifaqeer/oit-dns}}
      \item {Implemented an exploratory ETL pipeline in PySpark to extract packet header features and cluster IoT network traffic.}
%      
      \item {\textbf{Internet of Unpatched Things:} Tested multiple IoT devices in the lab and exposed their vulnerabilities.
      		\link{https://www.cs.princeton.edu/news/citp-stitching-internet-things}}
      \item {Found that PixStar's digital photo frame was susceptible to eavesdroppers and fails to encrypt photographs, and the Nest thermostat exposed private location information of nearest weather stations to the ISP (now patched).}
%      		
      \item {\textbf{Broadband traffic analysis:} Analyzed usage behavior of customers offered higher speed broadband without their knowledge.}
      \item {Found that difference in traffic demand was higher for moderate users as compared to high-volume subscribers.
      		\link{https://github.com/shahifaqeer/comcast-analysis}}
      \item {Presented at PAM (Mar 2016), CableLabs (Jul 2016), and FCC (Oct 2016).
      		\link{https://www.youtube.com/watch?v=9QiovWtGtq0}}
%
      \item {\textbf{Censor Planet:} Developed scripts to analyze censorship data collected throughout the world using TCP SYN-ACK attacks.}
      \item {Implemented multiprocessing python code to filter, analyze, and recombine large traces of server responses to TCP packets.}
      \item{Used time-series analysis to conclude whether there is server-side blocking, client-side blocking, or no firewall.
      		\link{https://github.com/shahifaqeer/spooky-analyzer}}
%
%      \item {TA for ... }
%        \item {Provisioned an easily managable hybrid infrastructure(Amazon AWS + On-premise) utilizing IaC(Infrastructure as Code) tools like Ansible, Packer and Terraform.}
%        \item {Built fully automated CI/CD pipelines on CircleCI for containerized applications using Docker, AWS ECR and Rancher.}
%        \item {Designed an overall service architecture and pipelines of the Machine Learning based Fashion Tagging API SaaS product with the micro-services architecture.}
%        \item {Implemented several API microservices in Node.js Koa and in the serverless AWS Lambda functions.}
%        \item {Deployed a centralized logging environment(ELK, Filebeat, CloudWatch, S3) which gather log data from docker containers and AWS resources.}
%        \item {Deployed a centralized monitoring environment(Grafana, InfluxDB, CollectD) which gather system metrics as well as docker run-time metrics.}
      \end{cvitems}
    }

%---------------------------------------------------------
  \cventry
    {Graduate Research Assistant} % Job title
    {Georgia Institute of Technology} % Organization
    {Atlanta, GA, USA} % Location
    {Aug. 2012 - Dec. 2014} % Date(s)
    {
      \begin{cvitems} % Description(s) of tasks/responsibilities
      \item {Advisers: Prof. Nick Feamster (University of Chicago) and Prof. Renata Teixeira (INRIA Paris)}
%      
      \item {\textbf{Home network analysis:} Analyzed active and passive network traces from multiple homes to study network availability and reliability in various countries, popular devices and infrastructure in homes, and traffic usage with time.
      		\link{https://www.lincs.fr/events/peeking-behind-the-nat-an-empirical-study-of-home-networks/}}
      \item {Discovered most home traffic is exchanged to a small number of domains, and home network usage differs based on the device.}
      \item {Analyzed prevalence and persistence of traceroutes to a variety of Internet destinations from the perspective of access points.}
%      \item {Examined route persistence and prevalence. Presented talk at AIMS, CAIDA.}
%     		
      \item {\textbf{SAZO:} Built and deployed a blacklist based malware identification and notification system on the home router.
      		\link{https://sarthakgrover.github.io/project/sazo/}}
      \item {Used bloom-filters to index malicious IPs and perform quick look-ups on packet headers before redirecting traffic to DPI boxes.}
%      
      \item {\textbf{Facade:} Built and deployed an HTTP pluggable transport protocol to avoid censorship and detection for Tor.
      		\link{https://github.com/shahifaqeer/facade}}
      \item {Responsibilities included parallelizing code, implementing framing buffers, and unit testing the system in a team of 6 students.}
%      
      \item {\textbf{QoS control using SDN:} Identified application and used SDNs to program appropriate rate shaper and control network flow.}
      \item {Identified traffic using a DNS classifier and demonstrated ad-block for the whole home using FlowQoS at the access point.
     		\link{https://sarthakgrover.github.io/project/flowqos/}}
%      \item {Implemented RESTful API server for car rental booking application(CARPLAT in Google Play).}
%      \item {Built and deployed overall service infrastructure utilizing Docker container, CircleCI, and several AWS stack(Including EC2, ECS, Route 53, S3, CloudFront, RDS, ElastiCache, IAM), focusing on high-availability, fault tolerance, and auto-scaling.}
%      \item {Developed an easy-to-use Payment module which connects to major PG(Payment Gateway) companies in Korea.}
      \end{cvitems}
    }

%---------------------------------------------------------
  \cventry
    {Junior Research Fellow} % Job title
    {Indian Institute of Science} % Organization
    {Bangalore, INDIA} % Location
    {Apr. 2011 - Dec. 2011} % Date(s)
    {
      \begin{cvitems} % Description(s) of tasks/responsibilities
      \item {Adviser: Prof. Anurag Kumar (Director, IISc Bangalore).}
%      
      \item {WSNs for Societal Needs and Disaster Management: Prepared work-plan proposal for submission to the Department of Science and Technology (DST), Govt of India.}
      \item {Computed closed-formed expressions for network reliability for regular hexagonal network topology.}
      \item {Evaluated information-theoretic bounds on network reliability for random hybrid network topology.}
%      \item {Studied tessellation algorithm for WSN drop and placement to ensure end-to-end connectivity in the wild.}
%      
%      \item {\textbf{SmartConnect: DIT-ASTEC WSN Project}: Deployment of industrial wireless sensor networks, project demonstrations, and experimental data analysis.}
      \end{cvitems}
    }
    
%to work on the government initiated Indo-
%Brazil WINSON project which targets wireless network design for societal needs in devel-
%oping countries. The problem at hand was to develop a complete cost-effective wireless solution
%for patient monitoring in hospitals. Mathematically, we needed to obtain an optimum relay
%node placement for high network reliability under QoS constraints. I inferred novel equivalence
%relations between hexagonal lattices of varying radii using Group Theory and proposed a tier-
%rank based routing protocol. To consider arbitrary node placement, I examined the cutting-edge
%works of Bacceli, Penrose, Franceschetti and Meester, and obtained the base station density
%for CSMA based random networks. Presently, in conjunction with other doctorate stu-
%dents, we are aiming to derive information theoretic models to provide best performance for
%in-building networks. Along with demanding extensive mathematical analysis my work also
%requires articulate presentation skill to coordinate our efforts with the other research groups. In
%conjunction with this research, I audited courses on Wireless Communication and Stochas-
%tic Processes & Queuing Theory, and as part of the SmartConnect project team, I also
%worked on industrial network deployment and experimentation.

%develop a complete cost-effective wireless solution for patient monitoring in
%hospitals. Mathematically, we needed to obtain an optimum hybrid network which provides high
%reliability under QoS constraints. Considering lattice networks, I inferred novel equivalence relations
%between hexagonal lattices of varying radii using Group Theory and proposed a tier-rank based
%routing protocol. To consider arbitrary node placement, I examined the cutting-edge works of Bacceli,
%Penrose, Franceschetti and Meester, and obtained the base station density for CSMA based random
%networks. Moreover, I have observed a surprising dependence of the minimum radio range for
%connectivity in hybrid wireless networks on the density of base stations in random geometric graphs.
%Presently, in conjunction with other doctorate students, I am studying this peculiar behavior to derive
%information theoretic models which provide best performance for in-building networks.
%---------------------------------------------------------
  \cventry
    {Graduate Research Assistant} % Job title
    {Indian Institute of Technology, Roorkee} % Organization
    {Roorkee, INDIA} % Location
    {Jul. 2009 -- Dec. 2010} % Date(s)
    {
      \begin{cvitems} % Description(s) of tasks/responsibilities
      \item {\textbf{Performance Evaluation of a Wireless Body Area Network:} Implemented cross layer protocol to auto-regressively predict PHY parameters and control MAC level queue for mobile nodes in Rayleigh fading environment. Modeled human body channel for intra BAN (on-body network). Comprehensive simulations on NS2 showed improved network throughput and lifetime.
      		\link{https://sarthakgrover.github.io/publication/performance-WBAN2010.pdf}}
      \item {\textbf{Implementation of a Soft Decision Decoder using Trellis on FPGA:} Implemented a real-time trellis decoder for BCH codes using VHDL on Xilinx ISE and configured it on FPGA.}
    \end{cvitems}
    }
    
%Using channel state prediction at MAC layer, I developed
%a Cross-Layer Protocol based on 802.11. Performing comprehensive simulation studies on
%NS2 network simulator, I demonstrated the improvements offered in energy consumption and
%scheduling. This work finds application in power efficient monitoring of athletes and mobile
%patients using on-body and in-body wireless sensors.    
    
%challenging problem of developing a cross
%layer protocol for WBANs. I utilized PHY layer information to predict the channel state at MAC layer
%and control the queue in IEEE 802.11 protocol. Performing comprehensive simulation studies on NS2
%network simulator, I demonstrated the improvements offered in energy consumption and scheduling.
%This work finds application in power efficient monitoring of athletes and mobile patients using on-
%body and in-body wireless sensors. I learnt a humble lesson when my results could not be confirmed
%on a practical test-bed, and it further strengthened my belief in validating firm theoretical arguments
%through experimental testing for application oriented research.
    
%---------------------------------------------------------
\end{cventries}

%%	\item \textbf{Internet of Things: Traffic Analysis, Device Fingerprinting, and Security:} Exposed privacy issues of various IoT devices through detailed traffic analysis in a lab environment; Analyzed IoT traffic at university router level and clustered devices based on traffic features; Proposed DNS-based identifier and malware traffic tracker for home networks with IoTs, and coded a test version in Spark.
%
%\head{Work experience}
%
%% write what exactly you did
%% write in points
%
%\begin{itemize}[label={}]
%
%	\item \textbf{Princeton University:} \emph{Research assistant} \hfill \textbf{Jan 2015 -- Sep 2017} \\
%	\vspace{-\topsep}
%	\begin{list2}
%	\item \textbf{Internet of Things: Traffic Analysis, Device Fingerprinting, and Security:} Exposed privacy issues of various IoT devices through detailed traffic analysis in a lab environment; Analyzed IoT traffic at university router level and clustered devices based on traffic features; Proposed DNS-based identifier and malware traffic tracker for home networks with IoTs, and coded a test version in Spark.
%	%TODO check doc with work done for better keywords
%	\item \textbf{xxx:}
%	\end{list2}	
%
%
%% SUMMER INTERNS SEPARATED OR IN WORK EX????
%%  at Paris, Comcast ???? No UMD		
